\documentclass[a4paper,10pt]{article}
\usepackage[latin1]{inputenc}
\usepackage[spanish]{babel}
\usepackage{graphicx}
\usepackage{verbatim}
\usepackage{moreverb}
\usepackage{amsmath}
\usepackage{amsfonts}
\usepackage{amssymb}
\usepackage{fancybox}
\usepackage{float}
\usepackage{fancyvrb}
\usepackage{color}

%opening
\title{Manual de usuario}
\author{}

\begin{document}
\maketitle
\begin{center}
\bigskip\bigskip\bigskip\bigskip
\Large Mini Sistema Operativo \\
\Large HAL 9000
\end{center}

\clearpage

\section{Comandos}
\par
\subsection{Help}
\par
Se utiliza para desplegar en pantalla la lista de comandos disponibles.\\
\includegraphics[scale=0.7]{ejemplo1.jpg}
\\
\subsection{Man}
\par
Recibe como argumento el nombre de otro programa y muestra la forma de usarlo y sus par�metros.\\
\includegraphics[scale=0.7]{ejemplo2.jpg}
\\
\subsection{Clear}
\par
Limpia la pantalla. Antes de usarlo: \\
\includegraphics[scale=0.7]{ejemplo3a.jpg}
Presionando la tecla 'Enter':\\
\includegraphics[scale=0.7]{ejemplo3b.jpg}
\\
\subsection{TTY}
\par
Usado con el par�metro '-s N' donde N es un n�mero entre 0 y 9, cambia a la terminal indicada por $N$ (tambi�n se pueden usar las teclas F1 a F9)\\
\includegraphics[scale=0.7]{ejemplo4a.jpg} \\
Presionando la tecla 'Enter':\\
\includegraphics[scale=0.7]{ejemplo4b.jpg}
\par
Con el par�metro '-l' cambia a la siguiente terminal y con el par�metro '-n' cambia a la anterior.\\
Usando '-ss A' cambia el s�mbolo de sistema por el texto A que no puede tener m�s de 15 caracteres. \\
\includegraphics[scale=0.7]{ejemplo4c.jpg} \\
Por �ltimo, con el par�metro '-c F B', cambia el color de la $font$ por F y el del $background$ por B. Los valores v�lidos para F y B 
van del 0 al 9.
\includegraphics[scale=0.7]{ejemplo4d.jpg}\\
\subsection{Echo}
\par
Imprime en pantalla los argumentos que recibe a continuaci�n de su invocaci�n.\\
\includegraphics[scale=0.7]{ejemplo5.jpg} \\
\subsection{Time}
\par
Imprime la hora, los minutos y los segundos.\\
\includegraphics[scale=0.7]{ejemplo6.jpg}
\\
\subsection{MKExc}
\par
Genera una de las 32 posibles excepciones (algunas de las cuales est�n reservadas). Recibe como argumento el n�mero de excepci�n, que puede ir 
de 0 a 31. \\
\includegraphics[scale=0.7]{ejemplo7.jpg}
\\
\subsection{Arnold}
\par 
Devuelve en pantalla frases aleatorias de las prosas de Arnold Alois Schwarzenegger, h�roe contempor�neo.\\
\includegraphics[scale=0.7]{ejemplo8.jpg}
\\
\subsection{CPUID}
\par
Imprime las caraceter�sticas del procesador.\\
\includegraphics[scale=0.7]{ejemplo9.jpg}
\\
\subsection{Bingo}
\par
Juego de bingo para dos personas.\\
\includegraphics[scale=0.7]{ejemplo10.jpg}
\\
\subsection{Reboot}
\par
Reinicia el sistema.\\
\subsection{History}
\par
Imprime el historial de comandos.\\
\includegraphics[scale=0.7]{ejemplo11.jpg}
\\
\end{document}

