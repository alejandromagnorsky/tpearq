\documentclass[a4paper,10pt]{article}
\usepackage[utf8]{inputenc}
\usepackage[spanish]{babel}
\usepackage{graphicx}
\usepackage{verbatim}
\usepackage{moreverb}
\setlength{\parindent}{0pt}
\setlength{\parskip}{10pt plus 6pt minus 4pt}
\usepackage{amsmath}
\usepackage{amsfonts}
\usepackage{amssymb}
\usepackage{fancybox}
\usepackage{color}


\begin{document}
\begin{center}
\textbf{\LARGE{Arquitecturas de Computadoras}}\\
\large{TP Especial}\par \noindent \newline \newline
\large{Mini Sistema operativo}
\end{center}
\par \noindent \newline \newline \newline \newline \newline \newline \newline
\newline \newline \newline \newline \newline
\large{\textbf{Integrantes}}\\
\indent \small{MAGNORSKY, Alejandro}\\
\indent \small{MATA SUÁREZ, Andrés}\\
\indent \small{MERCHANTE, Mariano}\newline
\par \noindent
\large{\textbf{Cátedra}}\\
\indent \small{VALLÉS, Santiago Raúl}\\
\indent \small{EL NOMBRE DEL NUEVO NO ME LO SE NI LO DICE IOL}\newline
\par \noindent
\large{\textbf{Fecha de entrega}}\\
\indent \small{Viernes 18 de junio de 2010}
\clearpage
\tableofcontents
\clearpage
\section{Inconvenientes surgidos - FALTA BOCHA**********}
\indent Dada la imposibilidad de uso de las librerías de C oficiales, en determinadas situaciones nos vimos obligados a
ajustar el diseño de nuestro sistema de acuerdo con la implementación de nuestra propia libreria de C. Tales percances
y sus resoluciones son expuestos a continuación:\\

\textbf{\large{Alocación en memoria}}\\
\indent Quizás este fue el problema con el que más frecuentemente tuvimos que lidiar. En otras circunstancias, el uso de la
función de reserva de espacio en memoria \texttt{malloc} de C nos hubiera permitido no muy dificilmente solucionar la cuestión.
Sin embargo, al no tener la bases suficientes como para implementar una función similar a la mencionada en nuestra librería,
en los casos que así lo dispusieron optamos por reservar cierta cantidad fija en memoria y sacrificar la
manipulación dinámica de ese espacio, mediante la declaración de vectores.\par \noindent \newline
\indent \textbf{Máxima cantidad de caracteres ingresados por el usuario}\par
La función \texttt{scanf} de \texttt{lib.c}, en un principio, solicita al usuario ingrese una string, invocando a la función
\texttt{getString} de la misma librería. Esta última función recibe un parámetro de tipo \texttt{char *} donde se guardará
cada caracter leído con \texttt{getchar}, dejando un último lugar para el caracter nulo \texttt{$\backslash$ 0)}. 
No devuelve nada; el parámetro
recibido es también de retorno.\par \noindent \newline
\indent Ahora bien, es tarea de la función invocadora (en este caso, \texttt{scanf}) pasarle a \texttt{getString} un \texttt{char *} con suficiente
espacio reservado para que la función se encargue de rellenarlo. Si el parámetro no tiene suficiente espacio para guardar
todo lo introducido por el usuario, se producirá un error de segmentación en memoría y el sistema se reiniciará.
\par \noindent \newline
\indent Y es aquí donde se toma la decisión. Con las constantes globales, \texttt{MAX\_ARGUMENTS} y \texttt{MAX\_ARGUMENT\_LENGTH},
se limita la cantidad de argumentos y caracteres por argumento que puede introducir el usuario, respectivamente:\par \noindent
\newline \newline \underline{en \texttt{shell.h}:}
\begin{verbatim}
. . .
#define MAX_ARGUMENT_LENGTH 32
#define MAX_ARGUMENTS 8
. . .
\end{verbatim}
\par Dicho esto, dentro de \texttt{scanf} y antes de la llamada a \texttt{getString}, se declara la variable que posteriormente será utilizada
como el parámetro de retorno que esta última función  requiere de la siguiente manera:\par \noindent \newline
\underline{en \texttt{libc.c}:}
\begin{verbatim}
. . .
#define MAX_USER_LENGTH MAX_ARGUMENTS*MAX_ARGUMENT_LENGTH
. . .
int scanf(const char * str, ...){
    . . .
    char strIn[MAX_USER_LENGTH+1];
    . . .
}
\end{verbatim}
\par Así, se reserva el suficiente espacio para que el usuario pueda meter \texttt{MAX\_USER\_LENGTH} caracteres, más uno para hacer
la string \texttt{null-terminated}.\par \noindent \newline
\clearpage
\section{Bibliografía consultada}
\begin{itemize}
\item \textbf{OSDev.org}\\
\textcolor{blue}{\underline{http://wiki.osdev.org/Main\_Page}}\\
\item \textbf{Programación del PIC}\\
\textbf{} \textcolor{blue}{\underline{http://www.beyondlogic.org/interrupts/interupt.htm\#6}}\\
\textbf{} \textcolor{blue}{\underline{http://www.osdever.net/tutorials/pdf/pic.pdf}}\\
\item \textbf{ASCIIs y ScanCodes}\\
\textbf{} \textcolor{blue}{\underline{http://www.win.tue.nl/~aeb/linux/kbd/scancodes-1.html}}\\
\textbf{} \textcolor{blue}{\underline{http://www.cdrummond.qc.ca/cegep/informat/Professeurs/Alain/files/ascii.htm}}\\
\textbf{} \textcolor{blue}{\underline{http://www.ee.bgu.ac.il/~microlab/MicroLab/Labs/ScanCodes.htm}}\\
\item \textbf{Manejo de excepciones}\\
\textbf{} \textcolor{blue}{\underline{http://pdos.csail.mit.edu/6.828/2008/readings/i386/s09\_08.htm}}\\
\item \textbf{Command bytes del teclado}\\
\textbf{Command bytes}\\
\textcolor{blue}{\underline{http://www.arl.wustl.edu/~lockwood/class/cs306/books/artofasm/Chapter\_20/CH20-2.html}}\\
\textbf{LEDs} \textcolor{blue}{\underline{http://wiki.osdev.org/PS2\_Keyboard\#Keyboard\_LEDs}}\\
\textbf{Reboot} \textcolor{blue}{\underline{http://wiki.osdev.org/Reboot}}\\
\end{itemize}
\clearpage

\section{Código fuente}
\subsection{Directorio src}
\large{\underline{\textbf{libc.c:}}}\\
\verbatimtabinput[4]{src/libc.c}

\end{document}
