\documentclass[12pt, a4paper, spanish]{report}
\usepackage[T1]{fontenc}
\usepackage{times}
\usepackage[spanish]{babel}
\usepackage[latin5]{inputenc}
\usepackage{textcomp}
\usepackage{listings}
\usepackage{color}
\definecolor{gray97}{gray}{.97}
\definecolor{gray75}{gray}{.75}
\definecolor{gray45}{gray}{.45}
\lstset{
     frame=Ltb,
     framerule=0pt,
     aboveskip=0.5cm,
     framextopmargin=3pt,
     framexbottommargin=3pt,
     framexleftmargin=0.4cm,
     framesep=0pt,
     rulesep=.4pt,
     backgroundcolor=\color{gray97},
     rulesepcolor=\color{black},
     stringstyle=\ttfamily,
     showstringspaces = false,
     basicstyle=\small\ttfamily,
     commentstyle=\color{blue},
     keywordstyle=\bfseries,
     numbers=left,
     numbersep=15pt,
     numberstyle=\tiny,
     numberfirstline = false,
     breaklines=true,
   }
\lstnewenvironment{listing}[1][]
   {\lstset{#1}\pagebreak[0]}{\pagebreak[0]}
\lstdefinestyle{C}{language=C}


\begin{document}
\begin{center}
\textbf{\LARGE{Arquitecturas de Computadoras}}\\
\large{TP Especial}\par \noindent \newline \newline
\large{Mini Sistema operativo}
\end{center}
\par \noindent \newline \newline \newline \newline \newline \newline \newline \newline \newline \newline \newline
\newline \newline \newline \newline \newline
\large{\textbf{Integrantes}}\\
\indent \small{MAGNORSKY, Alejandro}\\
\indent \small{MATA SU\'AREZ, Andr\'es}\\
\indent \small{MERCHANTE, Mariano}\newline
\par \noindent
\large{\textbf{C\'atedra}}\\
\indent \small{VALL\'ES, Santiago Ra\'ul}\\
\indent \small{EL NOMBRE DEL NUEVO NO ME LO SE NI LO DICE IOL}\newline
\par \noindent
\large{\textbf{Fecha de entrega}}\\
\indent \small{Viernes 18 de junio de 2010}

\tableofcontents

\chapter*{Inconvenientes surgidos - FALTA BOCHA**********}
\indent Dada la imposibilidad de uso de las librer\'ias de C oficiales, en determinadas situaciones nos vimos obligados a
ajustar el dise\~no de nuestro sistema de acuerdo con la implementaci\'on de nuestra propia libreria de C. Tales percances
y sus resoluciones son expuestos a continuaci\'on:\\

\textbf{\large{Alocaci\'on en memoria}}\\
\indent Quiz\'as este fue el problema con el que m\'as frecuentemente tuvimos que lidiar. En otras circunstancias, el uso de la
funci\'on de reserva de espacio en memoria \texttt{malloc} de C nos hubiera permitido no muy dificilmente solucionar la cuesti\'on.
Sin embargo, al no tener la bases suficientes como para implementar una funci\'on similar a la mencionada en nuestra librer\'ia,
en los casos que as\'i lo dispusieron optamos por reservar cierta cantidad fija en memoria y sacrificar la
manipulaci\'on din\'amica de ese espacio, mediante la declaraci\'on de vectores.\par \noindent \newline
\indent \textbf{M\'axima cantidad de caracteres ingresados por el usuario}\par
La funci\'on \texttt{scanf} de \texttt{lib.c}, en un principio, solicita al usuario ingrese una string, invocando a la funci\'on
\texttt{getString} de la misma librer\'ia. Esta \'ultima funci\'on recibe un par\'ametro de tipo \texttt{char *} donde se guardar\'a
cada caracter le\'ido con \texttt{getchar}, dejando un \'ultimo lugar para el caracter nulo (\texttt{\textquotesingle \textbackslash0\textquotesingle}). No devuelve nada; el par\'ametro
recibido es tambi\'en de retorno.\par \noindent \newline
\indent Ahora bien, es tarea de la funci\'on invocadora (en este caso, \texttt{scanf}) pasarle a \texttt{getString} un \texttt{char *} con suficiente
espacio reservado para que la funci\'on se encargue de rellenarlo. Si el par\'ametro no tiene suficiente espacio para guardar
todo lo introducido por el usuario, se producir\'a un error de segmentaci\'on en memor\'ia y el sistema se reiniciar\'a.
\par \noindent \newline
\indent Y es aqu\'i donde se toma la decisi\'on. Con las constantes globales, \texttt{MAX\_ARGUMENTS} y \texttt{MAX\_ARGUMENT\_LENGTH}, 
se limita la cantidad de argumentos y caracteres por argumento que puede introducir el usuario, respectivamente:\par \noindent
\newline \newline \underline{en \texttt{shell.h}:}
\begin{lstlisting}[style=C]
. . .
#define MAX_ARGUMENT_LENGTH 32
#define MAX_ARGUMENTS 8
. . .
\end{lstlisting}
\par Dicho esto, dentro de \texttt{scanf} y antes de la llamada a \texttt{getString}, se declara la variable que posteriormente ser\'a utilizada
como el par\'ametro de retorno que esta \'ultima funci\'on requiere de la siguiente manera:\par \noindent \newline
\underline{en \texttt{libc.c}:}
\begin{lstlisting}[style=C]
. . .
#define MAX_USER_LENGTH MAX_ARGUMENTS*MAX_ARGUMENT_LENGTH
. . .
int scanf(const char * str, ...){
    . . .
    char strIn[MAX_USER_LENGTH+1];
    . . .
}
\end{lstlisting}
\par As\'i, se reserva el suficiente espacio para que el usuario pueda meter \texttt{MAX\_USER\_LENGTH} caracteres, m\'as uno para hacer
la string \texttt{null-terminated}.\par \noindent \newline

\chapter*{Bibliograf\'ia consultada}
\begin{itemize}
 \item \textbf{OSDev.org}\\
	\textcolor{blue}{\underline{http://wiki.osdev.org/Main\_Page}}\\
 \item \textbf{Programaci\'on del PIC}\\
	\textbf{}  \textcolor{blue}{\underline{http://www.beyondlogic.org/interrupts/interupt.htm\#6}}\\
	\textbf{}  \textcolor{blue}{\underline{http://www.osdever.net/tutorials/pdf/pic.pdf}}\\
 \item \textbf{ASCIIs y ScanCodes}\\
	\textbf{}  \textcolor{blue}{\underline{http://www.win.tue.nl/~aeb/linux/kbd/scancodes-1.html}}\\
	\textbf{}  \textcolor{blue}{\underline{http://www.cdrummond.qc.ca/cegep/informat/Professeurs/Alain/files/ascii.htm}}\\
	\textbf{}  \textcolor{blue}{\underline{http://www.ee.bgu.ac.il/~microlab/MicroLab/Labs/ScanCodes.htm}}\\
 \item \textbf{Manejo de excepciones}\\
	\textbf{}  \textcolor{blue}{\underline{http://pdos.csail.mit.edu/6.828/2008/readings/i386/s09\_08.htm}}\\
 \item \textbf{Command bytes del teclado}\\
	\textbf{Command bytes}\\
	\textcolor{blue}{\underline{http://www.arl.wustl.edu/~lockwood/class/cs306/books/artofasm/Chapter\_20/CH20-2.html}}\\
	\textbf{LEDs}  \textcolor{blue}{\underline{http://wiki.osdev.org/PS2\_Keyboard\#Keyboard\_LEDs}}\\
	\textbf{Reboot}  \textcolor{blue}{\underline{http://wiki.osdev.org/Reboot}}\\
\end{itemize}

\chapter*{C\'odigo fuente}
\clearpage
\section*{Directorio src}
\large{\underline{\textbf{libc.c:}}}\\
\lstset{style=C}
\lstinputlisting{src/libc.c}

\end{document}
