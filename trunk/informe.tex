\documentclass[12pt, a4paper, spanish]{report}
\usepackage[spanish]{babel}
\usepackage[latin5]{inputenc}

\begin{document}
\begin{center}
\textbf{\LARGE{Arquitecturas de Computadoras}}\\
\large{TP Especial}\par \noindent \newline \newline
\large{Mini Sistema operativo}
\end{center}
\par \noindent \newline \newline \newline \newline \newline \newline \newline \newline \newline \newline \newline
\newline \newline \newline \newline \newline
\large{\textbf{Integrantes}}\\
\indent \small{MAGNORSKY, Alejandro}\\
\indent \small{MATA SU\'AREZ, Andr\'es}\\
\indent \small{MERCHANTE, Mariano}\newline
\par \noindent
\large{\textbf{C\'atedra}}\\
\indent \small{VALL\'ES, Santiago Ra\'ul}\\
\indent \small{EL NOMBRE DEL NUEVO NO ME LO SE NI LO DICE IOL}\newline
\par \noindent
\large{\textbf{Fecha de entrega}}\\
\indent \small{Viernes 18 de junio de 2010}

\tableofcontents

\chapter*{Inconvenientes surgidos - FALTA BOCHA************************************************}
\indent Dada la inexistencia de las librer\'ias de C oficiales, en varias situaciones en la construcci\'on del sistema nos vimos
obligados a ajustar el dise\~no de acuerdo a la implementaci\'on de nuestra propia libreria de C. Tales percances y sus resoluciones
son expuestos a continuaci\'on.\\

\textbf{Alocaci\'on en memoria}\\
\indent Durante la implementaci\'on de la funci\'on 'scanf', se present\'o el problema de la cantidad de caracteres que un usuario
del sistema operativo pueda introducir en consola. En otras circunstancias, esta cuesti\'on se hubiera resuelto no muy dif\'icilmente
con el uso de la funci\'on 'malloc' de C,\par
\indent Sin embargo, al no tener las bases necesarias para poder realizar una implementaci\'on de dicha funci\'on en nuestra
librer\'ia, optamos por fijar 2 valores: la cantidad m\'axima de argumentos que un usuario puede introducir (contando el nombre
del comando), y la cantidad m\'axima de caracteres que pueden tener dichos argumentos.\\
\indent Tales valores se encuentra definidos en 'sell.h', en la carpeta src:\\
\indent  \indent define MAX\_ARGUMENT\_LENGTH 32\\
\indent  \indent define MAX\_ARGUMENTS 8


%\section*{Cantidad de argumentos en l\'inea de comandos}
Dada la inexistencia de las librer\'ias de C correspondientes necesarias para el manejo de memoria din\'amica, en varias
ocasiones nos vimos forzados a tomar decisiones de dise\~no con respecto a c\'omo tratar la alocaci\'on de espacio en memoria.\\
\chapter*{Bibliograf\'ia consultada}
\par
\chapter*{C\'odigo impreso}
\par

\end{document}
